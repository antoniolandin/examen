\documentclass[a4paper,12pt]{article}
\usepackage{amsmath}
\usepackage{amsfonts}
\usepackage{amssymb}
\usepackage{geometry}
\usepackage{fancyhdr}
\usepackage{multicol}

% Page layout
\geometry{margin=1in}

% Header and footer
\pagestyle{fancy}
\fancyhf{}
\fancyhead[L]{Examen Estructuras Algebraicas}
\fancyhead[C]{MAIS}
\fancyhead[R]{Noviembre 05, 2024}
\fancyfoot[C]{\thepage}

\begin{document}

\begin{center}
    \textbf{\Large Estructuras 2024/25. Examen parcial} \\
    \textbf{Profesor: Georgy Nuzhdin. Fecha: 05.11.2024} \\
\end{center}

\section*{Teoremas}
\begin{enumerate}
    \item (0.3) En níngun grupo puede haber dos elementos neutros distintos.

    \item (0.5) Cualquier grupo de orden primo es cíclico.

    \item (0.7) Cualquier grupo es isomorfo a un subgrupo de algún grupo simétrico.

    \item (0.7) Las clases laterales izquierdas o coinciden o no tienen intersección.

    \item (1.0) $G / Kerf \cong Imf(f)$, siendo $f:G\to G'$ un homomorfismo.
\end{enumerate}

\section*{Ejercicios}

\begin{enumerate}
    \item Averigua si es semigrupo, monoide o grupo el conjunto de todos los subconjuntos del conjunto $A$ con la operación:

    \begin{enumerate}
        \item (0.2) La unión ($\cup$)
        \item (0.4) $a*b = (a \cup b) \setminus (a \cap b)$
    \end{enumerate}

    \item (0.5) Averigua cuanto es $(124)^{20} (23)^{33} (1234)^{-1}$, su orden y signatura

    \item Calcula los siguientes conmutadores:

    \begin{enumerate}
        \item (0.3) $[sr, sr^2]$ en $\Delta_{4}$
        \item (0.4) $[(123),(24)]$ en $S_{4}$
    \end{enumerate}

    \item (0.9) Estudia en cada caso si el morfismo $f(x) = x^3$ es homomorfismo, isomorfismo, endomorfismo o automorfismo

    \begin{enumerate}
        \item $\mathbb{Z}_3 \to \mathbb{Z}_3$
        \item $\mathbb{Z}_6 \to \mathbb{Z}_6$
        \item $\Delta_{3} \to \Delta_{3}$
    \end{enumerate}

    \item (1.0) Demuestra que no existe isomorfismo de grupo o constrúyelo para los siguientes cuatro grupos:
        $$
        (\mathbb{Q}, +) \quad (\mathbb{R}, +) \quad (\mathbb{R}^{+}, \times) \quad (\mathbb{C}^{*}, \times)
        $$

    \item (1.2) Comprueba que $\{e, sr\} \subset \Delta_{3}$ es un subgrupo. Indica sus clases laterales izquierdas (nombra sus elementos). ¿Forman un grupo? En el caso afirmativo, indica cuál es el subgrupo cociente.

    \item Busca el grupo de automorfismo para los siguientes grupos:

    \begin{enumerate}
        \item (0.4) $\mathbb{Z}_8$
        \item (0.4) $\mathbb{Z}_{11}$
        \item (1.0) $\Delta_{4}$
    \end{enumerate}

\item (2.0) Busca los subgrupos normales del grupo de cuaterniones $Q_8$ y analiza los grupos cocientes. ¿A qué grupos son isomorfos?

\end{enumerate}
\end{document}
